%>>>>
\documentclass[a4paper,10pt]{article}
%\documentclass[11pt,a4paper,sans]{article}

% Inclua o pacote xcolor com a opção dvipsnames
\usepackage[dvipsnames]{xcolor}

\usepackage[utf8]{inputenc}
%\usepackage{geometry}
\usepackage[scale=0.85]{geometry}
\geometry{a4paper, margin=30pt}
\usepackage{parskip}
\usepackage{enumitem}
\usepackage{hyperref}
\usepackage{titlesec}
%\usepackage{xcolor}
\usepackage{setspace}
\usepackage{nopageno}

\usepackage{helvet} % Carrega a fonte Helvetica
\renewcommand{\familydefault}{\sfdefault} % Define a fonte sans-serif como padrão
\usepackage[T1]{fontenc} % Codificação da fonte

% variaveis
% Definir cores personalizadas
%\definecolor{txtGeral}{RGB}{30,30,30} % Defina sua cor personalizada (verde floresta)
\definecolor{corHuge}{HTML}{275dad} % Titulo
\definecolor{corLarge}{HTML}{666666} % 
%\definecolor{corSection}{RGB}{10,102,194} % Skills
\definecolor{corSection}{HTML}{0857a6} % {RGB}{8, 87, 166}
\definecolor{corSubSection}{HTML}{0857a6} %{RGB}{8, 87, 166}
\definecolor{black}{RGB}{0,0,0}
%\definecolor{txtGeral}{HTML}{666666} % Defina sua cor personalizada (verde floresta)
\definecolor{txtGeral}{HTML}{484848} % Defina sua cor personalizada (verde floresta)

%7ab7ff
%666666
%275dad
%1e1e1e

\newcommand{\linkedin}{https://www.linkedin.com/in/sebastian-gonzalez-1253b912b}
\newcommand{\mobile}{+55~(19)~98801-1185}
\newcommand{\email}{segodimo@gmail.com}

% Defina a cor global do texto
\AtBeginDocument{\color{txtGeral}}

% Definir estilos de título
\titlespacing*{\section}{0mm}{6mm}{0mm}
%\titleformat{\section}{\Large\color{corSection}}{}{0mm}{}[\titlerule]
\titleformat{\section}{\Large\color{corSection}}{}{0mm}{}

% Definir margens de seção
\titlespacing*{\subsection}{0mm}{6mm}{0mm}
\titleformat{\subsection}{\large\bfseries\color{corSubSection}}{}{0mm}{}

%\usepackage[colorlinks=true, linkcolor=blue, urlcolor=red, citecolor=green]{hyperref}
\hypersetup{ colorlinks=true, urlcolor=corHuge, linkcolor=corHuge, citecolor=corHuge, pdfnewwindow=true }

%\titlespacing*{\section}
%  {\small\color{corSection}} % Formatação do título
%  {0pt}      % Espaçamento à esquerda
%  {10mm}      % Espaçamento acima do título
%  {2mm}     % Espaçamento abaixo do título (ajuste conforme necessário)
%\titleformat{\section}
%  {\large\bfseries\color{corSection}} % Formatação do título
%  {}                                  % Espaçamento do rótulo (label) do título
%  {0pt}                               % Espaçamento entre o rótulo e o corpo do título
%  {}                                  % Antes do título
%  [\titlerule\vspace{5pt}]            % Depois do título, com espaço extra

\onehalfspacing % Definindo espaçamento de um e meio globalmente


\begin{document}

\noindent % Evita a indentação da primeira linha


\hfill\textcolor{corSection}{\href{\linkedin}{\linkedin}} 
\begin{minipage}[t]{0.7\textwidth}
  \raggedright % Alinha o texto à esquerda
  \vspace{6mm} % Ajuste o espaçamento superior como desejado
  \Huge{\textcolor{corHuge}{Sebastian Gonzalez}} \\
  \vspace{2mm} % Ajuste o espaçamento superior como desejado
  \large{\textcolor{corLarge}{Desenvolvedor Full-Stack}} \textbar\ 
  \large{\textcolor{corLarge}{Node}} \textbar\
  \large{\textcolor{corLarge}{React}} \textbar\
  \large{\textcolor{corLarge}{Python}} \textbar\
  \large{\textcolor{corLarge}{Java}}
  %\large{\textcolor{corLarge}{Angular}}
  % \large{\textcolor{corLarge}{AWS}} \textbar\
\end{minipage}%
\begin{minipage}[t]{0.3\textwidth}
  \raggedleft % Alinha o texto à direita
  \setlength{\parskip}{0pt} % Ajuste o espaçamento entre linhas como desejado
  \textcolor{corSection}{\href{mailto:\email}{\email}} \\
  \textcolor{corSection}{\mobile} 

\end{minipage}

%\vspace{1mm}
\vspace{5mm}

Desenvolvedor com mais de 8 anos como Full-Stack.
Tenho experiencia no desenvolvimento de interfaces e microsserviços e integrações,
Trabalhei para clientes como SKY, DirecTV-GO
e Open Finance Brasil.

Eu fiz parte do desenvolvimento da arquitetura do fluxo de criação de conta da DirecTV-GO
usada em vários países de latino América,
também já atuei em governança de dados com bases de cerca de 500 mil contas em produção.

%====================================================================================================
\section*{Experiência Profissional}
\noindent\makebox[\linewidth]{\rule{\linewidth}{0.1mm}\textcolor{corLarge}{}}

\subsection*{Accenture Brasil}
\textcolor{corSubSection}{\emph{Analista de Desenvolvimento de Sistemas}}
\hfill \textcolor{corSubSection}{Set. de 2020 - Mai. de 2024}

%Minhas principais responsabilidades e atribuições incluíram:

\vspace{2mm} \textcolor{corSubSection}{\bfseries{SKY:}}
Trabalhei com a integração de microsserviços e interfaces focadas em prospectar clientes.
Atuei no desenvolvimento de uma ferramenta inovadora para conversão de clientes SKY
usando modelos estatísticos para gerar propostas personalizadas de planos,
visando aumentar a taxa de conversão de clientes prospectivos. 

Desenvolvi microsserviços para processar dados sensíveis que os clientes
desejavam tratar dentro da base da SKY para cumprir com a LGPD.

\vspace{2mm} \textcolor{corSubSection}{\bfseries{DirecTV-GO:}}
Trabalhei com arquitetura de microsserviços baseada em eventos
para desenhar o fluxo de criação de contas de usuários
da a DTVGO em toda a América Latina.

Integrei o fluxo de criação de conta para que os usuários do México
pudessem abrir as contas usando a operadora telefônica AT\&T.

Atuei no desenvolvimento de diversas funcionalidades na AWS,
incluindo o uso de filas SNS, SQS (com Retry e DLQ)
e aplicações distribuídas com Step Functions.

Atuei em governança de dados escrevendo scripts
para automatizar processos de detecção e ajuste das contas
assim como na aplicação de campanhas,
lidando com bases de relatórios armazenados em S3 que continham aproximadamente
800 mil linhas em produção para uma base de aproximadamente 500 mil contas.

\vspace{2mm} \textcolor{corSubSection}{\bfseries{Open Finance Brasil:}}
Usando Next.Js eu fiz parte da construção da interface dos dashboards
da Open Finance Brasil (iniciativa do Banco Central)
usados para compartilhar dados financeiros entre instituições.
Também desenvolvi microsserviços backend para fazer as integrações BFB e BFF.

\subsection*{Capela Software}
\textcolor{corSubSection}{\emph{Analista de Sistemas}}
\hfill\textcolor{corSubSection}{Set. de 2019 - Jun. de 2020}

Trabalhei no departamento de desenvolvimento como programador pleno Full-Satck
na criação de um sistema de análise Fiscal e Tributária
baseado na estrutura do SPED da Receita Federal,
usado para auditar e comparar todas as informações fiscais,
tributárias e financeiras de empresas,
validando os dados vindos de diversas fontes.

Trabalhei no desenvolvimento da interface com dashboards
criando componentes em React e usando bibliotecas como
Bicharts, Recharts, AntV para visualizar dados.

Atuei também no backend desenvolvendo e integrando microsserviços com Node e Express,
usamos Java Spring Batch e Spring Boot para a importação de grandes masas de dados.

\clearpage
%================================================================

\subsection*{Requestia}
\textcolor{corSubSection}{\emph{Programador Full Stack}}
\hfill \textcolor{corSubSection}{Abr. de 2018 - Mar. de 2019}

Trabalhei como desenvolvedor no departamento de marketing
em conjunto com o departamento de desenvolvimento;
trabalhei com microsserviços focados ao marketing
e também na automação de campanhas.

Fiz treinamentos e cursos sobre ITSM(ITIL),
trabalhei também com Centro de serviços Compartilhados,
Notificação e escalamento baseados em SLA's.

\subsection*{REC Produçoes Interativas}
\textcolor{corSubSection}{\emph{Programador Full Stack}}
\hfill \textcolor{corSubSection}{Set. de 2015 - Abr. de 2018}

Programador no desenvolvimento de sites como "Rec Soluções Interativas",
"Posclique", "Soy Cho", "Rabbit Studio", "Amálgama", "Villa Karry",
"Bússola Interna", "Obsucra Store" e "Interface Dropbox" entre outros,
\href{https://www.youtube.com/watch?v=V_ahsuHgIoE}{segue aqui um reel de alguns projetos}.

%\subsection*{Art MX}
%\textcolor{corSubSection}{\emph{Programador Full Stack}}
%\hfill \textcolor{corSubSection}{\hfill Jun. de 2016 - Mar. de 2017}
%
%Programador de sites como "UAI Brasil", "http://itacolomi.siteimobiliario.net/", "https://www.canaldoserralheiro.com.br/" entre outros


%>>>>
\section*{Skills}
\noindent\makebox[\linewidth]{\rule{\linewidth}{0.1mm}\textcolor{corLarge}{}}

\textcolor{corSubSection}{Arquitetura de Software:}
Arquitetura de microsserviços, 
padrões como MVC, MVVM, ports and adapters/Hexagonal

Clean Code, SOLID, Design Patterns.
%Linting, Formatting, Linters, Prettier, Eslint, Stylelint,

\textcolor{corSubSection}{Padrões de Design:}
APIs REST, Serverless, GraphQL, SOAP,
Arquitetura Orientada a Eventos (lambda ).

\textcolor{corSubSection}{Javascript:}
TypeScript, Node, Express, Hapi,
Stack MERN, MEAN (React, Angular),
control de estados como Redux (Saga, Thunk), Hooks,
Babel, Webpack.

\textcolor{corSubSection}{Cloud services:}
Notificações SNS,
filas SQS, (FIFO, Retry e DLQ),
concorrência usando Batching de Mensagens e Auto Scaling,
Integração com Step Functions,
microsserviços Serverless ou kubernets,
armazenamento em S3,
gerenciamento e escalabilidade de instancias com EC2,
monitoramento com CloudWatch.

\textcolor{corSubSection}{CI/CD/DevOps tools:}
Pipelines em Jenkins,
Cobertura de testes,
Controle de Versão GIT (Bitbucket, GitHub, GitLab).

\textcolor{corSubSection}{Bancos de Dados:}
SQL relacionais como
MySQL, PostgreSQL, DynamoDB, e SQLite, RDS(AWS),
NoSQL como
MongoDB, Redis;
ORM como JPA, Hibernate.


\textcolor{corSubSection}{Testes:}
Testes unitários em Js usando Mocha, Chai, e Sinon, Jest; Jasmine,
Em Java, JUnit, Spring Test, Mockito,
Cobertura, Jacoco e Codecov,
automação de teste, Selenium, Postman,
SonarQubem;

%testes unitarios (cobertura)
%testes de integração (conexões sem mocks)
%testes end to end (testes como um usuário com Cypress)
%testes de carga (Artillery)

\textcolor{corSubSection}{Containerização e Orquestração:}
Ambientes isolados no Docker, Experiência com Containerização e Orquestração,
Virtualizar servidores e deploy em contenedores Docker na AWS,

\textcolor{corSubSection}{Python:}
Pandas, APIs Django | Flask, Scraping e Crawling com Beautifulsoup,
automação de request via API, Scrapy e Selenium,
Processamento de linguagem natural (PNL) com spaCy,
UI com Tkintee, ORM, SQLAlchemy, Pymongo,

\textcolor{corSubSection}{Java:}
Spring Boot, desenvolvimento de microsserviços, APIs REST e aplicações web,
injeção de Dependência, integração e manipulação de dados utilizando
JPA, Hibernate,

Build e Gerenciamento de Dependências, Maven, Gradle

Spring Batch com fluxos de processamento em lote,
particionamento,
chunk processing,
agendamento de tarefas, retry policies,
e controle de transações.

\textcolor{corSubSection}{Metodologia ágeis:}
Experiência trabalhando na metodologia Scrum, Kanban, plataformas ágeis como Jira, Requestia

%\textcolor{corSubSection}{Desenvolvimento Android:} Android SDK, Kotlin, Java, Android Jetpack Arquiteturas MVVM e MVC

%================================================================
\clearpage
\section*{Formação Acadêmica}
\noindent\makebox[\linewidth]{\rule{\linewidth}{0.1mm}\textcolor{corLarge}{}}

\subsection*{Universidade Estadual de Campinas}
\textcolor{corSubSection}{\emph{Pós Graduação em Marketing Organizacional}}
\hfill \textcolor{corSubSection}{2018 - 2020}

Fiz disciplinas que tem sido muito uteis na minha carreira como
desenvolvimento de produtos e serviços,
pesquisa de mercado e administração entre outras.

\subsection*{Universidade Estadual de Campinas}
\textcolor{corSubSection}{\emph{Cursos Estudante Espacial Faculdade de Engenharia Elétrica e de Computação (FEEC)}}
\hfill \textcolor{corSubSection}{2014 - 2015}

\begin{itemize}
    \item Machine learning, Neural network, Genetic Algorithms(GAs), Bees Algorithms, Ant Colony Optimisation
    \item Sistemas de Cognição Artificial e Tópicos em Engenharia de Computação VI
    \item Big Data - Visualização e Gerência de Informação
\end{itemize}

\subsection*{Universidad de San Buenaventura}
\textcolor{corSubSection}{\emph{Extensão Arte e Comunicação Visual}}
\hfill \textcolor{corSubSection}{2008 - 2009}

\subsection*{Universidad de San Buenaventura}
\textcolor{corSubSection}{\emph{Graduação Engenharia de Som}}
\hfill \textcolor{corSubSection}{2004 - 2009}








%\section*{Projetos}
%\subsection*{Nome do Projeto}
%\emph{Descrição do Projeto}\\
%Detalhes sobre o projeto, tecnologias usadas e resultados obtidos.

\end{document}

